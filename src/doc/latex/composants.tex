%!TEX root = ./sujet-projet.tex

%
% Spécification des composants
%
\chapter{Spécification des composants}\label{chapter:composants}

\section{Introduction}

	\subsection{Objectif}
Préciser les objectifs de ce chapitre. 

	\subsection{Organisation du chapitre}
Cette section décrit le contenu du reste du chapitre  et explique comment le document est organisé.
\section{Description des composants}

Établir les frontières du système.

Division du système en composants.

Décrire le comportement souhaité des composants.

\subsection{Le composant A}

\begin{figure}[htbp]
	\centering
%		\includegraphics[scale=1]{file}
	\caption{Le composant A et ses interfaces}
	\label{fig:label}
\end{figure}

\section{Interactions}

Décrivez, à haut-niveau, la collaboration entre les composants majeurs, pendant la réalisation des cas d'utilisation. 
Utilisez des interactions, c'est à dire, des diagrammes de séquence et des diagrammes de communication, pour décrire les cas nominaux et extra-nominaux des cas d'utilisation.


\emph{Ne vous limitez pas à une seule interaction par cas d'utilisation}

\subsection{Cas d'utilisation UC1}

\begin{figure}[htbp]
	\centering
		% \includegraphics[scale=1]{file}
	\caption{Interaction \--- Création d'une tâche}
	\label{fig:label}
\end{figure}

\begin{figure}[htbp]
	\centering
		% \includegraphics[scale=1]{file}
	\caption{Interaction \--- Création d'une tâche répétitive}
	\label{fig:label}
\end{figure}

\section{Spécification des interfaces}

	\subsection{Interface A}
	
	Présentation de l'interface en UML (ou HUTN). Description du comportement de chaque opération. Spécification éventuelle des pré-conditions en OCL.
	
\begin{itemize}
	\item \code{getTasksForDate(Date) : Task[*]} \\
	Retrouve un ensemble de tâches pour une date donnée.
\begin{ocl}
getTasksForDate(Date) : Task[*]
pre: 
-- Le package \emph{listings} n'est pas compatible avec UTF8.
-- Donc, n'utilisez pas d'accent.
\end{ocl}
	
	\item \code{removeAllTalks()} \\
	Efface toutes les tâches, éteint le serveur et débranche son câble de la prise électrique.
	
	\item \code{getTaskForNamde(String, Number) : Task} \\
	Retrouve une tâche à partir de son nom et son numéro de téléphone.
\end{itemize}	
	\subsection{Interface B}
	
	\subsection{Interface C}

\section{Spécification des types utilisés}

Spécifiez ici les types utilisés par les interfaces (seulement ceux qui ne font pas partie des types de base UML).

\subsection{Task}

	La classe \code{Task} représente une Tâche. Elle possède les attributs suivants:
	\begin{description}
		\item[name]: le nom de la tâche. 
		\item[isActive]: détermine si la tâche est activée ou non.  
	\end{description}

\begin{figure}[htbp]
	\centering
%		\includegraphics[scale=1]{task}
	\caption{La classe Task}
	\label{fig:taks}
\end{figure}

\section{Conclusion}
